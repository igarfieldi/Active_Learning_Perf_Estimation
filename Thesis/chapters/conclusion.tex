\ifgerman{\chapter{Fazit und weiterführende Arbeit}}{\chapter{Conclusion and Future Work}}
\label{conclusion}

In the scope of this work we introduced four estimators making use of both cross-validation and bootstrap techniques as well as curve fitting to estimate a classifier's performance. As part of the evaluation framework they were tested on four different datasets with three active learners and three function models and compared to both k-fold cross-validation as well as -632+ bootstrapping with regard to their error against holdout accuracy.

The results of this are somewhat mixed. On the one side, using not only the current classifier state for performance estimation appears to be a promising approach. Fitting a single learning curve as well as simulating multiple possible ways the classifier's training set could have been build both show bias and error spread similar to that of k-fold cross-validation and .632+ bootstrapping. On the other side, this comes at a hefty price; the times necessary to perform the estimation is order of magnitudes larger. The evaluation also showed that a complex function model may not be the best for fitting a learning curve, seeing as the simple linear model accomplished the lowest bias.

In the light of active learning, the path-based estimator also offers a potential advantage. In this work, it was assumed that every one of these paths to the current classifier state would be equally likely. However, using a non-random active learner like PAL, this is not true. Taking the probability of each path into account as well as using bootstrapping instead of cross-validation may lead to an improvement in classifier performance and should be looked into in further research.

A large impact also has the curve fitting itself. Using statistical weights to place higher importance on certain fitting points turned out to be a good addition, but not to the method it was expected to. Investigation into why the weights improved the path-based linear method is suggested; this in combination with a better function model could further decrease its bias.
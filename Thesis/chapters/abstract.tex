\ifgerman{\chapter*{Inhaltsangabe}}{\chapter*{Abstract}}

In dieser Arbeit wird die Fragestellung beleuchtet, ob eine Kombination von Kurvenregression und Kreuzvalidierung die Leistung eines Klassifikators bei Benutzung verschiedener aktiver Lerner ohne systematische Abweichungen schätzen kann, besonders wenn noch nicht viele Trainingsinstanzen vorliegen. Zu diesem Zwecke stellen wir vier verschiedene Verfahren auf Basis von \textit{leave-one-out Kreuzvalidierung} sowie verschiedenen Funktionsmodellen vor. Mit in Betracht gezogen werden auch die Streuung der Schätzungen sowie die benötigte Rechenzeit. In simulierten Tests werden die Kombinationen für verschiedene Datensätze mit bereits erprobten Schätzern verglichen.

In this work we evaluate the question if a combination of curve fitting and cross-validation produces an estimator without systematic error capable of assessing a classifier's performance, especially with few purchased training instances. For this purpose we present four methods based on \textit{leave-one-out cross-validation} as well as different function models. Also of interest are the spread of the error made as well as the necessary computation time. The methods are tested and compared against well-tried estimators.